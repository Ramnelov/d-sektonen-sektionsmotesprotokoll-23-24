\documentclass[../testamente_sekreterare_21-22.tex]{subfiles}

\begin{document}
G-Suite (kallas numera egentligen Google Workspace, men alla säger fortfarande G-Suite) är paketet av Google:s alla tjänster för företag och organisationer och är vad vi använder för våra konton, mailadresser, dokument osv. Det är sekreterarens uppgift att hålla koll på sektionens alla konton och se till att de som behöver ett konto får ett. G-Suite kostar egentligen för företag, men de har en version för ideella organisationer, som oss. Vi får då obegränsat med konton som vi kan skapa och använda oss av.

Dessa konton hanteras på sidan \url{admin.google.com}, som bara vissa har tillgång till. Du som sekreterare är super-admin och har fulla rättigheter att göra allt som går att göra på G-Suite. Mitt år (och året innan) så har även ordförande, vice ordförande och webmaster varit super-admin. Det finns även vissa andra nivåer, men används inte så mycket. Det skulle kanske kunna utnyttjas inte så mycket. Ordförande och vice är mest för backup, och webmaster kan behöva tillgång för att ändra i olika domäninställningar och liknande.

Mailadressen \texttt{sekreterare@d-sektionen.se} är \textquote{catch-all} för \texttt{@d-sektionen.se}. Detta betyder att du kommer få alla mail som går till adresser som ingen annan har. Det händer att man får mail där det stavats fel eller där någon försäljare bara skickar till någon adress i hopp om att nå fram.

\subsection{Users, groups och organizational units}
Detta är tre av de viktigaste delarna av G-Suite och är det man arbetar mest med. En user är ett konto men kan också ha flera alias kopplat till konto. I och med att GDPR blev en grej som vi behövde följa så bytte vi från att en roll har ett konto som går i arv till att ha att varje person som behöver en mail får en personlig mail (se mer i \cref{sec:gsuite_gdpr}). Till detta mail kopplar vi alias som personen behöver för sin roll. T.ex. har mitt konto \texttt{max.mogren@d-sektionen.se} som \textquote{primary email}. Till detta har jag bland annat \texttt{sekreterare@d-sektionen.se}. Skulle jag ha flera roller så kan jag få fler alias som jag kan använda.

Flera användare kan vara samlade i en Group. Det finns en grupp för de flesta utskott, och ganska många där till. Varje grupp har en mail associerad till sig som kan användas för att skicka mail till samtliga i den gruppen, t.ex. har vi i styrelsen \texttt{styret@d-sektionen.se}. Vissa av dessa är gamla och har använts för enstaka tillfällen eller är helt enkelt utdaterade, så detta skulle behövas städas. \todo{Behövs dessa städas?} Det går även att ha grupper i andra grupper, vilket kan förenkla ibland. När du skapar konton så behöver du då alltså se till att hen också läggs till i rätt grupp/grupper.

Nästa koncept är \textquote{organizational units}. Man kan beskriva detta som undergrupper till D-sektionens huvudsakliga G-Suite. Idag finns endast LINK och STABEN som separata OU:s, men det finns en tanke om att ta in D-Group som en OU \todo{Är D-Group en OU?}. Fördelen med en OU är att det gör det lättare att gruppera konton samt att ge vettiga admin-rättigheter. Bland annat så kan det vara bra att t.ex. vissa personer i LINK har rättigheter att skapa användare i LINK-OU:n och samma för STABEN, så att de kan vara mer självgående. Men prata med berörda parter så löser det sig! Värt att notera att admins i dessa OU:s inte har rätt att ändra andra admins.

Berörd partner för STABEN har varit kassören. Berörd partner för LINK har varit projektledare och/eller IT-ansvarig.

\subsection{Personliga konton och GDPR}
\label{sec:gsuite_gdpr}
Det var Anna Montelius, sekreteraren två år innan mig, som gjorde ganska stora ändringar i hur sektionen hanterar sina Google-konton för att de ska vara kompatibla med GDPR. Tidigare så har kontona gått i arv, alltså att jag hade fått ta över inlogget till kontot som har primary mail \texttt{sekreterare@d-sektionen.se}. I och med detta så får jag också tillgång till alla dokument, mail, kalenderevent m.m som alla tidigare personer med kontot har haft. Man hade kunnat lösa detta genom att be samtliga städa sina konton, men det är sällan folk gör det ordentligt. De ansåg därför att det är bättre att varje person har ett personligt konto och att man delar relevanta filer (utan personlig information!) när man ska lämna över posten.

Det finns dock två undantag till detta. Det är kontot \texttt{instagram@d-sektionen.se} \textbf{Note: } är ej säker på hur denna mail ser ut idag. Dessa konton är kopplade till instagramen respektive många andra konton som behövs för att tekniken ska fungera. Det är alltså okej att dessa konton går i arv, men be gärna ägaren av kontot att rensa dessa konton innan de lämnas över.

Jag gjorde så att alla fick en adress med \texttt{fornamn.efternamn@d-sektionen.se}, trots att de kommer använda en annan domän, t.ex. \texttt{@staben.info}, som huvudadress. De får då istället ett alias med den domän de då kommer använda huvudsakligen. Jag har märkt att det kan vara bra att förklara för de som får konton hur de gör för att skicka mail från olika alias. Google har en bra supportsida för hur man gör detta\footnote{\href{https://support.google.com/mail/answer/22370}{\texttt{support.google.com/mail/answer/22370}}}.


\subsection{Säkerhet}
Som du märker så har du väldigt mycket kontroll och makt med ditt G-Suite-konto. Detta betyder att det är extra extra viktigt att du har ett bra och säkert lösenord och använder two factor authentication för att skydda ditt konto. Detta gäller även speciellt de andra som har någon sorts admin-roll i G-Suite. Det är också bra att påminna samtliga som får ett konto om att ha ett bra lösenord. Det är också bra ur säkerhetsynpunkt att hålla G-Suite:en städad och se till att konton som inte längre används antingen suspendas eller tas bort.

Det finns även sidor och inställningar för säkerheten på D-sektionens G-Suite. Här finns information om rapporterade mail, intrångsförsök osv. Det händer sällan sådana saker, men kan vara bra att ha ett öga på det och se om man behöver göra.


\subsection{Tips}
Jag vill dela med mig vad som har varit svårt för min del och saker jag hade kunnat göra bättre angående Google Workspace. Det första som kan vara otroligt svårt att hålla koll på är alla olika alias som finns på sektionen. En utav de första sakerna du kommer behöva göra är att byta alias från det föregående verksamhetsåret till det nästa. Inför nolle-p är det rätt många konton som behöver uppdateras och kan vara krångligt att hålla koll på. Jag gjorde så att jag tog upp två flikar på personen från föregående år och den personen jag vill sätta alias på, och kopierade rakt av de alias som fanns. Jag kopierade även dokumentet `Sektionsaktiva' och checkade av alla så jag inte missade någon. Jag tror däremot det är en bättre idé att låta varje enskilt utskott lämna in ett dokument med poster och alias till dem posterna. Det underlättar dels för dig men även för utskotten så de vet vilka alias de har. Det är även viktigt att vara tydlig med det tidigare verksamhetsåret att man ska flytta alias, och det bör man skriva i Slack under sommaren när du påbörjar ditt verksamhetsår. Det finns ibland dem som vill spara sin alias lite till, för det mesta kassörerna som fortfarande får fakturor. Jag ska försöka hjälpa dig med den här övergången eftersom det var lite överväldigande för min del då man precis börjat använda admin-konsolen. Så det är något vi får kommunicera inför sommaren!
\end{document}