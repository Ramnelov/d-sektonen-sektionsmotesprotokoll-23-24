\documentclass[../testamente_sekreterare_21-22.tex]{subfiles}

\begin{document}
Hej! Var roligt att du har blivit D-Sektionens sekreterare! Tanken är att jag här ska gå igenom och förklara huvuduppdragen som jag som sekreterare har haft under året och vad man bör tänka på. Nedan kommer först lite om mig och vilka uppdrag jag har haft, och så kommer jag sedan gå in på viktiga delar mer noggrant.

Min förhoppning är att jag har fått med det mesta och lyckas summera posten. Troligt är dock att jag har missat vissa delar och att du kommer behöva ställa vissa frågor till mig. Vissa delar är dessutom säkert skrivna i lite väl mycket detalj, min tanke har varit att hellre vara lite övertydlig istället för att hoppa över någon del som du kanske inte förstår. Mitt tips är att börja läsa igenom detta testamente för att åtminstone få en överblick över vad som behövs göras och ungefär hur det fungerar.

Mycket av detta dokument är nästan direkt avskrivet från tidigare års sekreterares testamente. Jag ser till att också dela dessa filer med dig så också kan kolla i dem om du skulle behöva!

Lycka till med ditt år!

\subsection{Om mig}
Jag heter Max Mogren och kommer ursprungligen från Trosa, söder om Stockholm. Pluggade tidigare juridik i Uppsala men bytte spår helt och hämnade här i Linköping och har trivits otroligt bra sen dess.

Mitt tidigare engagemang är att jag var Chief i D-Group 20/21 och att gå med i styrelsen som sekreterare var rätt annorlunda från något jag hade gjort tidigare. Det jag märkt av dock är att det inte krävs några särskilda erfarenheter för posten, men i vissa fall är det viktigt att vara rätt nogrann och dubbelkolla en del saker.

Jag tycker det har varit superkul att vara en del av styret och hoppas att du också kommer tycka det!

Om du har några frågor så är det bara att kontakta mig! Jag vet att mycket kan vara lite klurigt, speciellt i början, så hjälper gärna till! Skickar med mina kontaktuppgifter så att du har dom.

\begin{table}[H]
  \centering
  \begin{tabular}{ll}
    E-mail (Privat) & \href{mailto://maxuno@live.se}{\texttt{maxuno@live.se}}                                  \\[1ex]
    E-mail (LiU)    & \href{mailto://maxmo027@student.liu.se}{\texttt{maxmo027@student.liu.se}}                \\[1ex]
    Telefonnummer   & \href{tel://+46708822634}{070 88 22 634}                                                 \\[1ex]
    Facebook-profil & \href{https://www.facebook.com/max.mogren.1/}{\texttt{facebook.com/max.mogren.1}}        \\[1ex]
  \end{tabular}
\end{table}

\subsection{Översikt över mina uppdrag}
Det står egentligen inte så mycket om rollen sekreterare i våra stadgar och reglemente. Mer exakt så står det följande:
\begin{displayquote}[Stadgarna]
  \itshape
  Det åligger sekreteraren att:
  \begin{itemize}
    \item vid sektionens sammanträden föra protokoll
    \item ansvara för sektionens stadgar
  \end{itemize}
\end{displayquote}
\begin{displayquote}[Reglementet]
  \itshape
  \begin{itemize}
    \item ... och sektionssekreterare är ständig sekreterare
    för protokollförda sammanträden. (Angående Kansliet, se §4)
  \end{itemize}
\end{displayquote}

Denna del är ganska självklar. Det är dock inte endast detta jag har gjort under mitt år. Framförallt har jag också ansvarat sektionens alla mailaddresser och Google-konton. Utöver detta så har det varit \textquote{vanligt} styrelsearbete.

\end{document}