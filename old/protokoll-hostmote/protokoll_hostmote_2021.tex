\documentclass{datateknologsektionen-document}

\usepackage{subfiles}

\newcommand{\ind}{\hspace*{2em}}
\newcommand{\motetbeslutar}{\textbf{Mötet beslutar}}
\newcommand{\att}{\\\ind\textbf{att}}
\newcommand{\rostlangd}[1]{Röstlängden är \textbf{#1}.}

\title{Protokoll | D-sektionens höstmöte 2021}
\date{2021-11-07}

\begin{document}
% ----------------------
% Title page
% ----------------------
\hspace{0pt}
\vfill
\begin{center}
    \Huge\textbf{Protokoll \\ D-sektionens höstmöte 2021}

    \huge Datateknologsektionen

    \large
    Organisationsnummer: 822002-1409

\end{center}
\vfill
{\large
    \textbf{Plats:} Campus Valla, C4 \\
    \textbf{Datum:} 2021-11-07 \\
    \textbf{Tid:} 13:00 \\
    \textbf{Firmatecknare} verksamhetsåret 21/22 (210701 - 220630) \\
    \ind Ordförande: Otto Heino \\
    \ind Kassör: Michelle Krejci 
}
\vfill
\hspace{0pt}
\pagebreak




% ----------------------
% Table of contents
% ----------------------
\tableofcontents
\pagebreak





% ----------------------
\section{Mötets öppnande}
Sektionsordförande \textbf{Otto Heino} förklarar mötet öppnat kl \textbf{13:39}.




% ----------------------
\section{Val av mötesfunktionärer}
\subsection{Mötesordförande}
\motetbeslutar\att{} välja \textbf{Otto Heino} till mötesordförande.

\subsection{Mötessekreterare}
\motetbeslutar\att{} välja \textbf{Max Mogren} till mötessekreterare.

\subsection{Justeringsperson, Rösträknare}
\motetbeslutar\att{} välja \textbf{William Thordson} och \textbf{Frida Kleringer} till justeringspersoner tillika rösträknare.




% ----------------------
\section{Adjungeringar och fastställande av röstlängden}
\motetbeslutar
\att{} fastställa röstlängden till \textbf{71}.





% ----------------------
\section{Fastställande av mötets stadgeenliga utlysande}
\textbf{Max Mogren} berättar hur och när mötet blivit utlyst.

\motetbeslutar
\att{} mötet utlysts stadgeenligt.



% ----------------------
\pagebreak
\section{Fastställande av föredragningslista}
\textbf{Otto Heino} presenterar ändringar från den preliminära till den slutgiltiga föredragningslistan.

\textbf{Otto Heino} föreslår att fastställa den slutgiltiga föredragningslistan.

\motetbeslutar\att{} fastställa föredragningslistan.


% ----------------------
\section{Rapport och meddelanden från styrelsen och kansliet}
\subfile{./sections/RoM.tex}


% ----------------------
\pagebreak
\section{Sektionens ekonomiska läge}
Sektionskassör \textbf{Michelle Krejci} presenterar på sektionsmötet.
D-sektionen har ett överskott på ungefär 320 000 kr. En del av överskottet ska förhoppningsvis gå till DÖMD för eventets 40 års jubileum. Donna har varit ekonomiskt ansvariga över rosa oktober vilket innebär att sektionens omsättning stiger.


% ----------------------
\section{Styrelsen informerar}
Styrelsen vill vara transparanta om att revisionsberättelser saknas. Tyvärr har sektionen inte haft revisorer för verksamhetsåret 20/21 och har därmed ingen revisionsberättelse för verksamhetsberättelserna. Ansvaret har iställets lagts på sektionskassören som har granskat verksamhetsberättelserna och de ekonomiska berättelserna. 

Styrelsen under verksamhetsåret 20/21 har dock haft god kontakt med ekonomiföreningen ELIN som vanligtvis ställer upp med revisorer. Därför kommer liknande situationer förhoppningsvis undvikas framöver.

% ----------------------
\section{Aktivitetsutskottet 20/21}
\subsection{Verksamhetsberättelse}
\textbf{Erik Halvarsson} presenterar verksamhetsberättelsen för Aktivitetsutskottet 20/21
\subsection{Ekonomisk berättelse}
\textbf{Fredrik Martinsson} presenterar den ekonomiska berättelsen för Aktivitetsutskottet 20/21

\motetbeslutar
\att{} ajournera mötet till kl 14:33.

\motetbeslutar
\att{} återuppta mötet kl 14:52.


% ----------------------
\pagebreak
\section{Arbetsmarknadsgruppen 20/21}
\subsection{Verksamhetsberättelse}
\textbf{Toibas Wang} presenterar verksamhetsberättelsen för Arbetsmarknadsgruppen 20/21.
\subsection{Ekonomisk berättelse}
\textbf{Erik Nordell} presenterar den ekonomiska berättelsen för Arbetsmarknadsgruppen 20/21.

% ----------------------
\section{D-Group 20/21}
\subsection{Verksamhetsberättelse}
\textbf{Max Mogren} presenterar verksamhetsberättelsen för D-Group 20/21.
\subsection{Ekonomisk berättelse}
\textbf{Filip Ekström} presenterar den ekonomiska berättelsen för D-Group 20/21.

% ----------------------
\section{STABEN 20/21}
\subsection{Verksamhetsberättelse}
\textbf{Adrian Byström} presenterar verksamhetsberättelsen för STABEN 20/21.
\subsection{Ekonomisk berättelse}
\textbf{Sophie Ryrberg} presenterar den ekonomiska berättelsen för STABEN 20/21.

% ----------------------
\section{Styrelsen 20/21}
\subsection{Verksamhetsberättelse}
\textbf{Carl Magnus Bruhner} presenterar verksamhetsberättelsen för styrelsen 20/21.
\subsection{Ekonomisk berättelse}
\textbf{Dag Jönsson} presenterar den ekonomiska berättelsen för styrelsen 20/21.

\section{Ekonomisk berättelse för verksamhetsåret 20/21}
\textbf{Dag Jönsson} presenterar den ekonomiska berättelsen för verksamhetsåret 20/21.

\section{Ansvarsfrihet för verksamhetsåret 20/21}
\textbf{Michelle Krejci} presenterar sin granskning av verksamhetsåret 20/21.

Styrelsen yrkar på att ansvarsbefria alla förtroendevalda under verksamhetsåret 20/21 förutom \textbf{Filip Ekström} med anledning av att det saknas underlag för bokföring.

Diskussion uppstår.

\textbf{Tobias Wang} föreslår att bordlägga ansvarsfriheten för \textbf{Filip Ekström} istället.

Styrelsen jämkar sig med förslaget från \textbf{Tobias Wang}.

\textbf{Dag Jönsson} yrkar på att ansvarsbefria alla förtroendevalda under verksamhetsåret 20/21 och neka ansvarsfrihet för \textbf{Filip Ekström}.

\textbf{Michelle Krejci} berättar att större delen av bokföringen för D-Group 20/21 har skett av \textbf{Dag Jönsson}.

En sektionsmedlem begär kontext och inblick i situationen.

\textbf{Filip Ekström} presenterar på sektionsmötet och förklarar problemet. Under verksamhetsåret har det skett misstag med insamlingen av kvitton på olika ställen. Det har lett till att en del av bokföringen fortfarande saknar underlag. Konversationer mellan \textbf{Filip Ekström} och dåvarande sektionskassör slutade med att \textbf{Dag Jönsson} gjorde en del av bokföring för D-Group 20/21 istället.

\textbf{Fabian Blom} berättar att Y-sektionens revisorer har granskat Arbetsmarknadsgruppens verksamhet och uppmärksammat en oegentlighet till följd av en intern fest.

\textbf{Erik Nordell} presenterar och ger mer kontext. 
Under en intern fest med Arbetsmarknadsgruppen gick ett brandlarm på grund av en rökmaskin. Dette ledde till en utryckning för falskt alarm. En avgift på 5 000 kr hamnade på ägaren av lokalen som betalades av Arbetsmarknadsgruppen. Detta är något som diskuterats mellan Y-styret, D-styret och Arbetsmarknadsgruppen som kom fram till att avgiften inte borde ha betalats av Arbetsmarknadsgruppen.

\textbf{Carl Magnus Bruhner} berättar att styrelsen 20/21 inte godkänner kostnaden. Däremot är det från styrelsens sida svårt att få en enskild person betala avgiften i detta fall. För att förhindra liknande händelser i framtiden har åtgärder gjorts i avtalet mellan styrelsen och Arbetsmarknadsgruppen.

Styrelsen föreslår att godkänna ansvarsfrihet för verksamhetsåret 20/21 och bordlägga ansvarsfriheten för \textbf{Filip Ekström}, samt notera att betalningen för utryckning av Arbetsmarknadsgruppen inte är acceptabel.

\motetbeslutar
\att{} välja \textbf{Richard Johansson} till mötesordförande.

Alla förtoendevalda under verksamhetsåret 20/21 lämnar rummet.

\motetbeslutar
\att{} fastställa röstlängden till 52.

\motetbeslutar
\att{} ansvarsbefria alla förtroendevalda under verksamhetsåret 20/21 men bordlägga beslutet om ansvarsfrihet för \textbf{Filip Ekström} samt notera att Arbetsmarknadsgruppens betalning av avgiften till följd av en intern fest inte är acceptabelt.

\motetbeslutar
\att{} fastställa röstlängden till 75.

\motetbeslutar
\att{} välja \textbf{Otto Heino} till mötesordförande.



% ----------------------
\section{Motioner och propositioner}
Inga inkomna motioner eller propositioner. 


% ----------------------
\pagebreak
\section{Personval}
\subsection{Fadderigeneral 2022}
Valberedningen nominerar \textbf{Casper Jensen} till fadderigeneral 2022.

\textbf{Casper Jensen} presenterar sig själv.\\
\textbf{Casper Jensen} svarar på frågor.\\
\textbf{Casper Jensen} lämnar mötet. 

\textbf{Malin Widén} läser upp Valberedningens motivering för \textbf{Casper Jensen}.

\textbf{Filip Jakobsson} yrkar på att valberedningen ska läsa upp deras rekryteringsprocess.

\textbf{Malin Widén} läser upp valberedningens rekryteringsprocess inför höstmötet 2021.

Disskussion uppstår.

\motetbeslutar
\att{} fastställa röstlängden till 70.

\motetbeslutar
\att{} välja \textbf{Casper Jensen} till fadderigeneral 2022.

\subsection{Fadderikassör 2022}
Valberedningen nominerar \textbf{Linnéa Ivansson} till fadderikassör 2022.\\
\textbf{Erik Nordell} nominerar sig själv.\\
\textbf{Jennifer Österman} nominerar sig själv.

\textbf{Erik Nordell} lämnar mötet.\\
\textbf{Jennifer Österman} lämnar mötet.

\textbf{Linnéa Ivansson} presenterar sig själv.\\
\textbf{Linnéa Ivansson} svarar på frågor.\\
\textbf{Linnéa Ivansson} lämnar mötet.

\textbf{Jennifer Österman} presenterar sig själv.\\
\textbf{Jennifer Österman} svarar på frågor.\\
\textbf{Jennifer Österman} lämnar mötet.

\textbf{Erik Nordell} presenterar sig själv.\\
\textbf{Erik Nordell} svarar på frågor.\\
\textbf{Erik Nordell} lämnar mötet.

\textbf{Simon Sandberg} läser upp valberedningens motivering för Linnéa Ivansson.

Diskussion uppstår.

\motetbeslutar
\att{} fastställa röstlängden till 70.

\motetbeslutar
\att{} välja \textbf{Linnéa Ivansson} till fadderikassör 2022.

\section{Medaljutdelning}
Medaljerna till tidigare års sektionsaktivas delas ut efter mötet. Kontakta intendent@d-sektionen.se vid frågor eller om du var frånvarande under mötet.

% ----------------------
\section{Övrigt}
Ingen fråga lyfts.

% ----------------------
\section{Mötets avslut}
Sektionsordförande \textbf{Otto Heino} förklarar mötet avslutat kl \textbf{18:06}.




% ----------------------
\pagebreak
{\Large\bfseries Underskrifter}

\vspace*{1.2cm}
\noindent\rule{8cm}{1pt}\\
Ordförande Otto Heino

\vspace*{1.2cm}
\noindent\rule{8cm}{1pt}\\
Sekreterare Max Mogren

\vspace*{1.2cm}
\noindent\rule{8cm}{1pt}\\
Justeringsperson William Thordson

\vspace*{1.2cm}
\noindent\rule{8cm}{1pt}\\
Justeringsperson Frida Kleringer


\end{document}
