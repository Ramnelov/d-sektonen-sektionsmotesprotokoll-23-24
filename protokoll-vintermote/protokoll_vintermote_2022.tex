\documentclass{datateknologsektionen-document}

\usepackage{subfiles}

\newcommand{\ind}{\hspace*{2em}}
\newcommand{\motetbeslutar}{\textbf{Mötet beslutar}}
\newcommand{\att}{\\\ind\textbf{att}}
\newcommand{\rostlangd}[1]{Röstlängden är \textbf{#1}.}

\title{Protokoll | D-sektionens vintermöte 2022}
\date{2022-01-31}

\begin{document}
% ----------------------
% Title page
% ----------------------
\hspace{0pt}
\vfill
\begin{center}
    \Huge\textbf{Protokoll \\ D-sektionens vintermöte 2022}

    \huge Datateknologsektionen

    \large
    Organisationsnummer: 822002-1409

\end{center}
\vfill
{\large
    \textbf{Plats:} Campus Valla, C4 \\
    \textbf{Datum:} 2022-01-31 \\
    \textbf{Tid:} 13:00 \\
    \textbf{Firmatecknare} verksamhetsåret 21/22 (210701 - 220630) \\
    \ind Ordförande: Otto Heino \\
    \ind Kassör: Michelle Krejci 
}
\vfill
\hspace{0pt}
\pagebreak




% ----------------------
% Table of contents
% ----------------------
\advance\cftsecnumwidth 0.5em\relax
\advance\cftsubsecindent 0.5em\relax
\advance\cftsubsecnumwidth 0.5em\relax
\tableofcontents
\pagebreak





% ----------------------
\section{Mötets öppnande}
Sektionsordförande \textbf{Otto Heino} förklarar mötet öppnat kl \textbf{17:34}.




% ----------------------
\section{Val av mötesfunktionärer}
\subsection{Mötesordförande}
\motetbeslutar\att{} välja \textbf{Otto Heino} till mötesordförande.

\subsection{Mötessekreterare}
\motetbeslutar\att{} välja \textbf{Max Mogren} till mötessekreterare.

\subsection{Justeringsperson, Rösträknare}
\motetbeslutar\att{} välja \textbf{William Thordson} och \textbf{John Enblom} till justeringspersoner tillika rösträknare.




% ----------------------
\section{Adjungeringar och fastställande av röstlängden}
\motetbeslutar
\att{} fastställa röstlängden till \textbf{78}.





% ----------------------
\section{Fastställande av mötets stadgeenliga utlysande}
\textbf{Max Mogren} berättar hur och när mötet blivit utlyst.

\motetbeslutar
\att{} mötet utlysts stadgeenligt.



% ----------------------
\pagebreak
\section{Fastställande av föredragningslista}
\textbf{Otto Heino} presenterar ändringar från den preliminära till den slutgiltiga föredragningslistan.

\textbf{Otto Heino} föreslår att fastställa den slutgiltiga föredragningslistan.

\motetbeslutar\att{} fastställa föredragningslistan.


% ----------------------
\section{Rapport och meddelanden från styrelsen och kansliet}
\subfile{./sections/RoM.tex}


% ----------------------
\section{Styrelsen informerar}
Sekreterare \textbf{Max Mogren} presenterar på sektionsmötet.
På det extrainsatta sektionsmötet 10 oktober presenterade \textbf{Dag Jönsson},\textbf{Michelle Krejci} och \textbf{Erik Halvarsson} en motion för omstrukturering av Aktivitetsutskottets ledning. På mötet antogs motionen men med styrelsens motionssvar. Tyvärr står det fel i protokollet från mötet. Just nu står det att ``Mötet beslutar att bifalla motionen i sin helhet.''. Det skall egentligen stå ``Mötet beslutar att jämka motionen med styrelsens motionssvar''.

Styrelsen vill därför uppmärksamma detta fel och notera det i det här protokollet eftersom det påverkar vad som skall stå i reglementet. För att bygga förtroende begärs bekräftelse från motionärerna Dag, Michelle och Erik, om de kan intyga att beslutet som togs på extramötet 10 oktober var att jämka motionen med styrelsens motionssvar.

Motionärerna \textbf{Dag Jönsson},\textbf{Michelle Krejci} och \textbf{Erik Halvarsson} intygar att beslutet som togs angående motionen var att jämka motionen med styrelsens motionssvar.


% ----------------------
\section{Sektionens ekonomiska läge}
Sektionskassör \textbf{Michelle Krejci} presenterar på sektionsmötet.
Näringslivsutskottet gör ett fortsatt bra arbete och det går väldigt bra för sektionen trots det nuvarande läget i samhället. Just nu plussar sektionen 340 000 kr.

% ----------------------
\pagebreak
\section{Budgetrevidering}
Sektionskassör \textbf{Michelle Krejci} presenterar på sektionsmötet.
Eftersom rambudgeten bestäms en gång per år kan den vara missvissande på grund av stora förändringar i budgeten. Syftet med budgetrevideringen är att försöka få rambudgeten reflekta verkligheten så riktigt som möjligt med de förändringar som görs under året.

Styrelsen har godkännt att D-Groups julfest samt sittningen Förspelet är med i budgeten. Styrelsen har även godkänt pengar till julmys och gåva till Y-sektionens femtioårs-jubileum.

Sektionskassör \textbf{Michelle Krejci} går igenom budgetrevideringen för varje enskilt utskott som påverkas.

D-Group kassör \textbf{Emma Cecilia Wahlund} går igenom den nya rambudgeten för D-Group.

STABEN kassör \textbf{Linnéa Ivansson} går igenom nya rambudgeten för STABEN.

Sektionskassör \textbf{Michelle Krejci} går igenom nya rambudgeten för styrelsen.

Sammanfattningsvis har dessa förändringar ökat underskottet för verksamhetsåret från -30 000 kr till -100 000 kr. Däremot gick sektionen med ett överskott på 380 000 kr förra verksamhetsåret vilket måste förbrukas vilket lämnar 280 000 kr för att täcka upp oväntande kostnader, varav 40 000 kr används för fonder.

\motetbeslutar
\att{} fastställa röstlängden till \textbf{78}.

\motetbeslutar
\att{} fastställa den nya rambudgeten.

% ----------------------
\section{Sektionens grafiska profil}
Vice-ordförande \textbf{Richard Johansson} presenterar på sektionsmötet.
På vårmötet 2021 diskuterades det om en ny grafisk profil till vintermötet 2022 med anledning av att den nuvarande loggan är svår att måla samt trycka på kläder.
Förslaget som presenterades under vårmötet 2021 kom plötsligt och sedan efter har styrelsen, Infoutskottets ordförande samt sektionens Art Director arbetat tillsammans mot en ny grafisk profil samt logga och har därför gjort en undersökning under december.

Art Director \textbf{Kimberley Andersson} presenterar resultaten av undersökningen.
Sammanfattningsvis är sektionsmedlemmar positivt inställda till en ny rund logga men anser att det behövs läggas mer tid och arbete på den nya sektionsloggan. En sektionsmedlem påpekar att den nuvarande loggan är relativt ny och därför är det inte särskilt brådskande med ett byte av sektionsloggan. Det är istället önskvärt att fortsätta lägga tillräckligt med arbete på den nya sektionsloggan så att den kommer användas långsiktigt.

Styrelsen har arbetat med att göra en planering för ny sektionslogga. På grund av resultaten från undersökningen läggs inget förslag fram om att godkänna loggan utan istället kommer arbetet med den grafiska profilen och nya loggan fortsätta med hjälp av planeringen styrelsen har gjort. Infoutskottet kommer fortsätta arbeta med en grafisk profil tillhörande loggan. 

En ny sektionslogga måste senast vara klar på vintermötet om det ska kunna appliceras på nästa verksamhetsår på grund av att STABEN beställer sektionsväskor i mars.


% ----------------------
\section{Motioner och propositioner}
Inga inkomna motioner eller propositioner. 

\motetbeslutar\att{} fastställa röstlängden till \textbf{88}.

\motetbeslutar\att{} ajournea mötet till kl 19:30.

\motetbeslutar\att{} fastställa röstlängden till \textbf{85}.

\motetbeslutar\att{} återuppta mötet kl 19:37.

\section{Presentation av STABEN}
STABEN 2022 presenterar sig sjävla och svarar på frågor.



% ----------------------
\pagebreak
\section{Personval}
\subfile{./sections/personval.tex}

\section{Medaljutdelning}
\textbf{Otto Heino} förklarar att medaljutdelning inte kommer kunna ske på dagens möte och att ärendet behandlas på nästa fysiska sektionsmöte.

% ----------------------
\section{Övrigt}
Ingen fråga lyfts.

% ----------------------
\section{Mötets avslut}
Sektionsordförande \textbf{Otto Heino} förklarar mötet avslutat kl \textbf{00:30}.




% ----------------------
\pagebreak
{\Large\bfseries Underskrifter}

\vspace*{1.2cm}
\noindent\rule{8cm}{1pt}\\
Ordförande Otto Heino

\vspace*{1.2cm}
\noindent\rule{8cm}{1pt}\\
Sekreterare Max Mogren

\vspace*{1.2cm}
\noindent\rule{8cm}{1pt}\\
Justeringsperson William Thordson

\vspace*{1.2cm}
\noindent\rule{8cm}{1pt}\\
Justeringsperson John Enblom


\end{document}
